% !TEX program = xelatex
\documentclass[14pt,a4paper]{extarticle}

% Базовые пакеты
\usepackage{fontspec}
\usepackage{expl3}

% Язык
\usepackage[russian]{babel}

% Шрифты
\setmainfont{Times New Roman}
\setsansfont{Arial}
\setmonofont{Courier New}

% Остальные пакеты без изменений...
\usepackage{geometry}
\geometry{
  a4paper,
  left=30mm,
  right=10mm,
  top=20mm,
  bottom=20mm
}
\usepackage{setspace}
\onehalfspacing
\usepackage{indentfirst}
\setlength{\parindent}{1.25cm}

% Заголовки
\usepackage{titlesec}
\titleformat{\section}{\normalfont\bfseries\large}{\thesection}{1em}{}
\titleformat{\subsection}{\normalfont\bfseries\normalsize}{\thesubsection}{1em}{}

% Нумерация страниц: арабскими цифрами по центру снизу
\usepackage{fancyhdr}
\pagestyle{fancy}
\fancyhf{}
\cfoot{\thepage}
\renewcommand{\headrulewidth}{0pt}

% Для листингов (SQL)
\usepackage{listings}
\usepackage{xcolor}
\lstset{
  basicstyle=\ttfamily\small,
  breaklines=true,
  frame=single,
  columns=fullflexible,
  keywordstyle=\color{blue}\bfseries,
  commentstyle=\color{gray},
  stringstyle=\color{teal},
  escapeinside={(*@}{@*)}
}

% Диаграммы ER (tikz)
\usepackage{tikz}
\usetikzlibrary{positioning,shapes.geometric,arrows.meta}

% Таблицы, картинки
\usepackage{booktabs}
\usepackage{longtable}
\usepackage{graphicx}
\usepackage{caption}

% Микротипография
\usepackage{microtype}

% Ссылки
\usepackage{hyperref}
\hypersetup{hidelinks}

% Библиография: biblatex-gost (ГОСТ Р 7.0.99-2018).
% Требует установленного пакета biblatex-gost в TeX Live/MiKTeX.
% Библиография - ИСПРАВЛЕННЫЕ НАСТРОЙКИ
\usepackage[%
  backend=biber,
  style=gost-numeric,
  sorting=none,
  language=auto,
  autolang=other,
  bibencoding=utf8
]{biblatex}
\addbibresource{references.bib}

\begin{document}

% ----------------- Титульный лист (не нумеруется, но учитывается) -----------------
\begin{titlepage}
  \thispagestyle{empty}
  \begin{center}
    \vspace*{2.5cm}
    {\large Федеральное государственное автономное образовательное учреждение высшего образования \\[3pt]
    «Сибирский федеральный университет» \\[6pt]
    Кафедра прикладной информатики в искусстве и интерактивных медиа \\[36pt]}
    {\LARGE\bfseries РЕФЕРАТ \\[12pt]}
    {\Large\bfseries Табличные данные и базы данных \\[24pt]}
    {\normalsize Тема: Табличные данные и базы данных (реляционные и нереляционные, СУБД) \\[12pt]}
    {\normalsize Выполнили: Тетерина П.А., Иванова С.Ю., Логинова Ю.В., Ситников А.В., Захаров И.М, Арсенян А.В., студенты 1 курса, направление «Прикладная информатика в искусстве и интерактивных медиа» \\[6pt]
    Руководитель: Нигматуллин И.Р., ст. преп. каф. ИТвКиКИ \\[36pt]}
    {\normalsize г. Красноярск \quad 2025}
  \end{center}
\end{titlepage}

% титульный лист считается, но номер не выводится; установить счетчик страниц так, чтобы
% следующая печатная страница имела номер 2 (титул = 1, невидимый)
\setcounter{page}{2}

% ----------------- Аннотация -----------------
\begin{center}
  \textbf{Аннотация}
\end{center}
Реферат посвящён принципам организации и хранения табличных данных в современных системах управления базами данных. Рассмотрены реляционные и нереляционные модели данных, основные функции СУБД, роль языка SQL, концепции целостности и надёжности, а также облачные подходы и инструменты для работы с базами данных. Работа подготовлена для бакалавриата (направление: прикладная информатика в искусстве и интерактивных медиа).

\medskip
\textbf{Ключевые слова:} база данных, реляционная модель, NoSQL, СУБД, SQL, целостность данных, надёжность, облачные СУБД, производительность.

\newpage
\tableofcontents
\newpage

% ----------------- Тело реферата -----------------
\section{Введение}
Современные цифровые проекты — интерактивные инсталляции, мультимедийные платформы и творческие приложения — опираются на надёжное хранение и обработку данных. Табличные представления остаются одним из наиболее удобных способов структурирования данных для приложений, требующих транзакционной точности и аналитики. В то же время требования к масштабируемости и гибкости формата данных стимулируют развитие альтернатив — NoSQL и облачных СУБД. Данная работа систематизирует ключевые принципы, сравнивает подходы, расширяет практические рекомендации по оптимизации и мониторингу и даёт инструкции для проектирования и эксплуатации систем хранения табличных данных с учётом специфики медиапроектов.

\section{Общие принципы организации и хранения данных в базах данных}
\subsection{Модель данных и уровни представления}
Модель данных определяет структуру, семантику и операции над данными. Традиционно выделяют три уровня представления: внешний (представления для пользователей и приложений), логический (структуры данных внутри СУБД: таблицы, отношения) и физический (как данные расположены на носителях: страницы, файлы, распределённые блоки). Ясное разделение уровней помогает абстрагировать бизнес-логику от аппаратных и СУБД-специфичных оптимизаций.

\subsection{Табличное представление и реляционная парадигма}
Таблица (отношение) — основной примитив реляционной модели. Каждая строка описывает экземпляр сущности, столбцы задают атрибуты с типами данных. Реляционная парадигма опирается на реляционную алгебру и обеспечивает формальную семантику операций: проекции, соединения, селекции и агрегирования. Это удобно для выражения бизнес-правил и построения аналитики. Нормализация до подходящих норм уменьшает аномалии обновления и избыточность.

\subsection{Физическое хранение: страницы, буфер, журнал}
На физическом уровне данные лежат в файлах, разбитых на страницы или блоки. Для уменьшения операций ввода-вывода СУБД используют буферный пул (кэш страниц в памяти). Для обеспечения надёжности и восстановления применяется журнальное логирование (WAL — write-ahead log). Контрольные точки (checkpoints) фиксируют состояние БД для ускорения восстановления.

\subsection{Row-store vs Column-store}
Строко-ориентированное хранение (row-store) предпочтительно при частых операциях вставки/обновления и при чтении полных записей (OLTP). Колонко-ориентированное хранение (column-store) эффективно при аналитических запросах (агрегации, сканирование столбцов), позволяет лучше сжимать данные. Некоторые СУБД и хранилища (например, гибридные движки) поддерживают оба режима или позволяют экспортировать данные из OLTP в OLAP-движок.

\subsection{Индексация и структуры ускорения доступа}
Индексы — ключевой инструмент ускорения выборок. B-деревья обеспечивают быстрый диапазонный поиск, хеш-индексы — быстрый точечный доступ, LSM-структуры (Log-Structured Merge Trees) применяются в системах с интенсивными записями. Инвертированные индексы используются для полнотекстового поиска. Индексация даёт выигрыш по чтению, но увеличивает издержки при записи и объём хранения — балансировка необходима.

\subsection{Оптимизация на физическом уровне}
Рассмотрены: правильный выбор размера страницы, настройка буферного пула, компрессия столбцов, хранение больших объектов (BLOB) — в СУБД или отдельно (object storage). В медиапроектах часто имеет смысл хранить контент (медиафайлы) в объектных хранилищах, а метаданные — в базе.

\section{Отличие реляционных и нереляционных моделей данных}
\subsection{Реляционная модель: сильные стороны и ограничения}
Реляционная модель даёт строгую схему, декларативность SQL и формальные механизмы целостности (PK, FK, CHECK). Ограничения: в распределённых сценариях традиционные реализации сложнее масштабировать по горизонтали; жёсткая схема усложняет быструю эволюцию структуры данных.

\subsection{NoSQL: основные типы и мотивы появления}
NoSQL-решения появились как ответ на требования веб-масштабируемости:
\begin{itemize}
  \item ключ-значение (Redis, DynamoDB) — простота и скорость;
  \item документные (MongoDB) — гибкая схема в формате JSON/BSON;
  \item ширококолончатые (Cassandra, HBase) — масштабируемость записи и распределённость;
  \item графовые (Neo4j) — модель для тесно связанных данных.
\end{itemize}
Они часто акцентируют доступность и масштабируемость, иногда в ущерб строгим транзакционным гарантиям; современные решения предлагают различные уровни согласованности.

\subsection{NewSQL и гибридные подходы}
NewSQL-проекты пытаются сочетать горизонтальную масштабируемость NoSQL с транзакционными гарантиями ACID. Практическая архитектура современных систем часто основана на polyglot persistence — сочетании реляционных и нереляционных технологий.

\subsection{Моделирование данных для медиапроектов}
В проектах интерактивного искусства и мультимедиа метаданные часто сложны: теги, временные интервалы, пространственные координаты, мультимодальные связи. Подходы:
\begin{itemize}
  \item хранить бинарные медиа в объектном хранилище (S3/MinIO), метаданные — в реляционной или документной БД;
  \item использовать документную модель для гибких метаданных (JSON), при этом поддерживать индексацию по ключевым атрибутам;
  \item применять графовые СУБД для сложных связей между медиа-объектами и взаимодействиями пользователей.
\end{itemize}

\section{Основные функции и возможности систем управления базами данных}
\subsection{CRUD, DDL и DML}
СУБД предоставляет операции создания, чтения, обновления и удаления (CRUD). DDL (Data Definition Language) — команды определения схемы; DML (Data Manipulation Language) — операции с данными. В медиа-приложениях важно правильно проектировать транзакции для последовательности действий (оплата, подтверждение, публикация).

\subsection{Транзакции, ACID и MVCC}
ACID и подходы реализации: блокировки, optimistic concurrency control, MVCC (многоверсионная конкуренция), что снижает конфликтность операций чтения и записи в высоконагруженных системах.

\subsection{Безопасность, доступ и аудит}
Аутентификация (LDAP/SSO), авторизация (ролевые модели), аудит операций, шифрование соединений (TLS) и хранения, маскирование секретных данных. Для медиапроектов — управление правами доступа на уровне объектов (ACL), DRM-интеграция.

\subsection{Репликация, шардирование, отказоустойчивость}
Репликация (синхронная/асинхронная), многомастерные конфигурации, шардирование и автоматическое распределение нагрузки. Рассмотрены сценарии failover, дедупликация данных и межрегиональная репликация для глобальных проектов.

\subsection{Инструменты резервного копирования и восстановления}
Резервные копии: полные, дифференциальные, инкрементные; горячие снапшоты; точечное восстановление по журналам. Практические рекомендации: регулярное тестирование восстановления, хранение бэкапов в другом регионе и в файловой системе, отдельное хранение больших медиа-файлов.

\section{Язык SQL как инструмент взаимодействия с базами данных}
\subsection{Роль SQL и стандарты}
SQL — декларативный язык, стандартизированный ISO. Его расширения включают оконные функции, CTE, JSON/JSONB-операции и процедурные языки (PL/pgSQL, T-SQL).

\subsection{Примеры запросов и объяснение планов}

Пример создания и аналитического запроса (см. ниже). Обсуждение EXPLAIN и EXPLAIN ANALYZE: как читать план, где ищутся узкие места (seq scan vs index scan), роль кардинальности и оценок оптимизатора.

\begin{lstlisting}[language=SQL, caption={Пример создания схемы и аналитики}]
CREATE TABLE customers (
  customer_id SERIAL PRIMARY KEY,
  name VARCHAR(200) NOT NULL,
  email VARCHAR(200) UNIQUE,
  registered TIMESTAMP DEFAULT CURRENT_TIMESTAMP
);

CREATE TABLE orders (
  order_id SERIAL PRIMARY KEY,
  customer_id INTEGER REFERENCES customers(customer_id),
  amount NUMERIC(10,2),
  order_date DATE
);

-- Аналитический запрос
EXPLAIN ANALYZE
SELECT c.name, COUNT(o.order_id) AS orders_count, SUM(o.amount) as total_amount
FROM customers c
LEFT JOIN orders o ON c.customer_id = o.customer_id
WHERE o.order_date BETWEEN '2024-01-01' AND '2024-12-31'
GROUP BY c.name
ORDER BY total_amount DESC;
\end{lstlisting}

Объяснение: план покажет узкие места. Типичные рекомендации: добавить индекс по order\_date, по customer\_id, рассмотреть партиционирование orders по дате.

\subsection{Материализованные представления и кеширование}
Материализованные представления (materialized views) — для предвычисленной аналитики. Кеширование на уровне приложения (Redis) позволяет снизить нагрузку на СУБД при частых повторяющихся запросах.

\section{Концепции целостности и надёжности в системах хранения данных}
\subsection{Типы целостности данных}
Сущностная (PK), ссылочная (FK) и семантическая (CHECK, триггеры). В медиапроектах — контроль целостности метаданных и ссылок на объекты в хранилище.

\subsection{Уровни изоляции, аномалии и стратегии}
Уровни изоляции (Read Uncommitted, Read Committed, Repeatable Read, Serializable). Пояснение аномалий и стратегий (оптимистичная/пессимистичная конкуренция).

\subsection{CAP, согласованность и распределённые транзакции}
CAP-теорема и практические стратегии: eventual consistency, causal consistency и strong consistency (в том числе глобальные транзакции и TrueTime в Spanner). Для медиапроектов допустимость eventual consistency часто выше, чем для платёжных систем.

\subsection{RPO, RTO, резервирование и тестирование}
Определение RPO и RTO. Практики: регулярные бэкапы, репликация, DR-планы, регулярное тестирование восстановления и учёт бизнес-требований.

\subsection{Соответствие требованиям и защита персональных данных}
GDPR и локальные требования: минимизация собираемых данных, шифрование, журналирование доступа, механизмы удаления данных по запросу (right to be forgotten). Внедрение процедур DPIA (Data Protection Impact Assessment).

\section{Облачные подходы и онлайн-инструменты для работы с базами данных}
\subsection{DBaaS и управляемые сервисы}
DBaaS снимает операционную нагрузку. Важно оценивать SLA, модели цен и возможности бэкапов/репликации.

\subsection{Облачные сервисы: сравнение и сценарии применения}
Краткое сравнение: Amazon RDS/Aurora (совместимость с PostgreSQL и MySQL), Google Cloud Spanner (глобально распределённая согласованность), Azure Cosmos DB (многомодельность), MongoDB Atlas, Amazon DynamoDB. Критерии выбора: согласованность, масштабируемость, стоимость и экосистема интеграций.

\subsection{ETL/ELT и архитектуры данных (Lake vs Warehouse)}
Различие: Data Lake (сырые данные, гибкость) vs Data Warehouse (структурированные данные для BI). Архитектуры: Lambda и Kappa для потоковой обработки.

\subsection{Инструменты мониторинга и DevOps-интеграция}
Prometheus, Grafana, pg\_stat\_statements (PostgreSQL), Percona Monitoring, Datadog — инструменты мониторинга. Flyway/Liquibase — версия схемы; Terraform — IaC; CI/CD — автоматическое тестирование миграций.

\section{Заключение}
Табличные данные и СУБД остаются фундаментом информационных систем, включая проекты в области искусства и интерактивных медиа. Реляционные СУБД обеспечивают строгую семантику и гарантии целостности, NoSQL и облачные решения — гибкость и масштабируемость. Производительность, мониторинг и соответствие регуляциям — ключевые практики. При проектировании важно учитывать специфику медиаконтента и комбинировать технологии в соответствии с требованиями по согласованности, доступности и стоимости.

\newpage
\printbibliography[title={Список использованных источников (по ГОСТ Р 7.0.99-2018)}]

\end{document}